\section{Related work}
\label{sec:related}

% \cite{becattini2017, hu2019, kukleva2019, han2017, li2017, pucci2023}
% \cite{becattini2017, pucci2023} - closest to there works
% \cite{kukleva2019, li2017} predicts progress, has a downstream task
% \cite{hu2019, han2017} joint prediction task. \cite{han2017} predicts in buckets

\noindent\textbf{Activity progress Prediction.} The task of progress prediction was formally introduced in \cite{becattini2017}. 
Because the progress of an activity is an easy-to-obtain self-supervision signal, it is often used as an auxiliary signal in a multi-task prediction problem, as in \cite{hu2019} to improve the performance of spatio-temporal action localisation networks. 
Progress prediction is also used as a pretext task for phase detection \cite{li2017}, or to create embeddings to perform unsupervised learning of action classes \cite{kukleva2019, vidalmata2020}. 
The progress prediction problem can also be modelled as a classification problem, choosing from $n$ bins each of size $1 / n$ as is done in \cite{han2017}. Based on the literature surveyed, of works done on progress prediction, only \cite{becattini2017, pucci2023} have progress prediction as their primary task. This work is also on the topic of progress prediction, but we do not propose our own progress prediction method. 
Instead, we consider the methods from \cite{becattini2017, kukleva2019} in our analysis and analyze their performance on the currently used datasets.

\smallskip\noindent\textbf{Remaining Duration.} A topic closely related to progress prediction is Remaining Duration (RD) prediction. 
While the goal of progress prediction is to predict the course of the activity as a percentage value in $[0,100\%]$, RD aims at predicting the remaining time $t$ in minutes or seconds. Previous work that researches the RD problem often does this in a surgical setting \cite{aksamentov2017, marafioti2021, twinanda2019, wang2013} and thus refers to it as the Remaining Surgery Duration (RSD) problem. Early methods work by pretraining a \textsl{ResNet-152} model to predict either the surgical phase \cite{aksamentov2017} or the surgery progress \cite{twinanda2019}, and then using the frame embeddings created from the ResNet-152 model in an LSTM block to perform RSD prediction. 
Building on top of this is the observation in \cite{marafioti2021} that predicting extra information such as surgeon skill, may be beneficial to do RSD prediction.
%which shows that by jointly predicting RSD, the surgeon's skill level as either a junior or senior surgeon and the surgical phase the RSD prediction improves. 
Finally, RSD can also be modelled in a way closer to progress. By dividing all RSD values by the highest possible RSD, the RSD can be predicted as a value between $0$ and $1$ \cite{wang2023}. Unlike these methods that model the passage of time as a decreasing remaining duration, we model it as an increasing progress value. 
We use \textsl{RSDNet} \cite{twinanda2019} in our analysis, as it performs both RSD and progress prediction.

\smallskip\noindent\textbf{Phase prediction.} If an action consistently consists of separate sub-tasks or phases of similar duration, then recognizing the current phase gives a good approximation of the progress.  
Previous work jointly performs phase-based progress prediction and surgical gesture recognition \cite{vanamsterdam2020}, jointly predicting the phase and the surgery tools \cite{twinanda2016}, or by using the embeddings in an LSTM to predict the surgical phase online \cite{yengera2018}. 
More recent work applies transformers to perform surgical phase recognition \cite{jamal2023, liu2023lovit}. In this work, we do not consider phase-prediction methods as they are an inaccurate proxy for progress. 
Furthermore, when activities are non-linear, phase prediction is no longer a good indicator of activity progress. 
Knowing which phase is happening may be useful as an extra signal, however we do not consider this, as it requires additional annotations.

\smallskip\noindent\textbf{Activity Completion.} The progress for each frame can be calculated using linear interpolation if the current activity time, $t$, the starting activity time, $t_\text{start}$, and the ending activity time, $t_\text{end}$, are available. Early work on this topic only predicts if an activity has been completed or not using an SVM \cite{heidarivincheh2016}. Follow-up work of Heidarivincheh \etal \cite{heidarivincheh2018} uses a CNN-LSTM architecture to predict the exact frame at which the activity is completed, \ie the activity completion moment. 
The detection of the activity completion moment is done in a supervised setting \cite{heidarivincheh2018}, where the exact frame at which the activity ends is annotated.
Alternatively, activity completion can be done in a weakly supervised setting where the only available annotation is if the activity has been completed or not \cite{heidarivincheh2019}. 
Although related to progress prediction, activity completion only aims at predicting the completion moment. In contrast, we focus on the more fine-grained targets of activity progression at every frame.

