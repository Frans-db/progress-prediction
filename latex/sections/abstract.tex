%%%%%%%%% ABSTRACT
\begin{abstract}
Activity progress prediction aims to estimate what percentage of an activity has been completed. 
Currently this is done with machine learning approaches, trained and evaluated on complicated and realistic video datasets. 
The videos in these datasets vary drastically in length and appearance.
And some of the activities have unanticipated developments, making activity progression difficult to estimate. 
In this work, we examine the results obtained by existing progress prediction methods on these datasets.
We find that current progress prediction methods seem not to extract useful visual information for the progress prediction task.
Therefore, these methods fail to exceed simple frame-counting baselines.
We design a precisely controlled dataset for activity progress prediction and on this synthetic dataset we show that the considered methods can make use of the visual information, when this directly relates to the progress prediction.
We conclude that the progress prediction task is ill-posed on the currently used real-world datasets.
Moreover, to fairly measure activity progression we advise to consider a, simple but effective, frame-counting baseline.
\end{abstract}
